\section{Related Work}
\label{sec:related}

% \vspace{-0.212mm}
\myparagraph{Confidential computing} We mainly focus on ARM TrustZone and Intel SGX TEEs for confidential computing~\cite{azure-sgx, ibm-sgx, alibaba-sgx, googleconfidentialvm}. 
%TrustZone is the key enabling technology for implementing TEEs on an Arm platform.
%TrustZone provides an isolated environment for security sensitive applications, and its mechanisms can be used to improve the security of the typical REE software stack, 
TrustZone is used in several workflows~\cite{pinto2019demystifying}, including monitoring kernel integrity~\cite{azab2014hypervision}, maintaining secrets~\cite{yun2019ginseng}, managing cryptographic material~\cite{li2014droidvault}, trusted language runtimes \cite{santos2014using}, or remote hardware management~\cite{brasser2016regulating}.
%
Likewise, Intel SGX has been extensively used to support many applications~\cite{baumann2014, arnautov2016, tsai2017}, 
%Our work builds on the SCONE framework.
%Intel SGX has become available in clouds~\cite{IBMCloudSGX, AzureSGX}, unleashing a plethora of services to be ported, 
including Web search~\cite{oblix-oakland-2018}, storage systems~\cite{avocado, rkt-io}, distributed systems~\cite{t-lease-socc-2020}, FaaS~\cite{clemmys}, networking~\cite{shieldbox, nsdi-safebricks-2018, rkt-io}, machine learning~\cite{secureTF-middleware, sgx-ml-usenix-security-2016}, and data analytics systems~\cite{sgx-pyspark,opaque-nsdi-2017,schuster2015vc3}. 
%In a similar vein of \project, to expand the possible use-cases of TEEs to new areas heterogenous computing architectures, 
Similarly, \project enables heterogeneous TEEs to interoperate, such as HETEE~\cite{zhu2020hetee} and Graviton~\cite{volos2018graviton}. In contrast, we present the first heterogeneous confidential computing framework for \csd. %in the context of query processing over \csd. 
%to be within a security domain. 
%In our work, we build a hetrogenous shielded execution framework for the NDP architecture. 
%Our system enables us to provide strong security properties for storage and query execution across  heterogenous TEE domains (SGX and TrustZone). 




\if 0
In our work, we leverage the secure environment provided by TrustZone to implement robust storage security primitives.
secure data processing, while distrusting system software (e.g., OS and Hypervisor).
A few examples include 
VC3~\cite{schuster2015vc3} which runs Hadoop in an SGX enclave to offer confidentiality and integrity guarantees, on MapReduce data processing workloads, 
TensorSCONE~\cite{kunkel2019tensorscone} which enables secure execution of machine learning computations in an untrusted infrastructure by leveraging SGX enclaves, and
SecureKeeper~\cite{brenner2016securekeeper} which enhances the ZooKeeper, a service for coordinating distributed applications, to protect user data only decrypting it within an SGX enclave.
In \project{} SGX is used to protect the \project{}-FQE and attestation services from potentially compromised high-privileged software.
\fi



% - cryptdb, seajoin: queries over encrypted data
% - monomi: split query exec over encrypted data
% - djoin: privacy preserving join across multiple dbs
% - enclavedb: in memory db with all properties guaranteed
% - tdb: hash of db in trusted storage
% - MiniCrypt~\cite{zheng-eurosys-2017},
% - speicher: KV store + LSM with secure storage and query ops
% - shieldstore: keep encrypted KV pairs outside enclave, use enclave only to protect them
% - SUNDR: nfs + detect unauthorized file modifications;
% - Plutus: encrypted file system; secure file sharing; crypto and key mgmt done by clients
% - jVPFS: journalling in VPFS split bw TCB and untrusted FS; reduce updates to the merkle tree; all sec props during unclean shutdown; journal hash in trusted, sealed memory; hash tree
% - sirius: secure file system; hash tree for freshness;
% - snad: secure NAS
% - maat: scalable security for distributed data
% - pcfs: policy enforcement of file accesses
% - strongbox: stream cipher; FDE
% - pesos and guardat: storage policy enforcement using a small software layer, at the storage layer
\myparagraph{Secure query processing systems}
%Secure storage and query processing systems are the bed rock to support secure data analytics in untrusted cloud environments. 
%For secure storage systems, a wide-range of systems have been proposed based on different TEEs with varying security guarantees and storage interfaces. 
Multiple secure storage and query processing systems have been proposed based on different TEEs, varying security guarantees, and storage interfaces~\cite{kim-eurosys-2019, pesos-eurosys-2018, ahmad2018obliviate, ahmad2018obliviate, bailleu2019Speicher,dickens-asplos-2018}. 
%Prominent TEE-based storage systems 
%Example systems include ShieldStore~\cite{kim-eurosys-2019}, Pesos~\cite{pesos-eurosys-2018}, OBLIVIATE~\cite{ahmad2018obliviate}, Speicher~\cite{bailleu2019Speicher}, and Strongbox~\cite{dickens-asplos-2018}.
%Likewise, 
Secure query processing systems~\cite{popa-sosp-2011, tu-vldb-2013, seabed-osdi-2016} describe approaches to execute queries directly on encrypted data, and propose various encryption schemes to make this possible. %DJoin~\cite{narayan-osdi-2012} describes a system to perform joins over multiple distributed databases in a privacy preserving manner. 
EnclaveDB~\cite{priebe-ieeesp-2018} presents the design of an in-memory database running inside an SGX enclave. %There has been extensive work on building systems that enable privacy preserving analytics and query processing. 
There exist work in secure query processing on privacy-preserving analytics~\cite{djoin-osdi-2012, emekci-icde-priv-preser-third-party, privatesql-vldb, PINQ-sigmod}, oblivious query processing~\cite{oblix-oakland-2018, 222619, 10.14778/3342263.3342641}, secure MPC~\cite{conclave-eurosys}.
In contrast, \project{} strives to ensure the security of data in transit, at rest, and processing across heterogeneous computational domains, enforcing access policies. %in transit, i.e query processing, and rest, i.e. untrusted persistent storage, across heterogeneous computational domains.

%DJoin~\cite{djoin-osdi-2012} describes a system to perform joins over multiple distributed databases in a privacy preserving manner. Emekci et al.~\cite{emekci-icde-priv-preser-third-party} describe techniques to perform privacy preserving query processing using third parties. They propose a hash based P2P system to select third parties to process queries, speeding up response times. Private SQL~\cite{privatesql-vldb} describes an end-to-end differentially private relational database system that enables answering if SQL queries with a fixed privacy loss across all queries. PINQ~\cite{PINQ-sigmod} describes a platform for performing data analytics in a privacy preserving manner. It provides a declarative programming language, in which all statements provide differential privacy and an execution environment that respects formal requirements of differential privacy. Conclave~\cite{conclave-eurosys} makes the observation that relational queries can maintain multi party communication's(MPC) end-to-end security guarantees without using MPC's cryptographic techniques for all operations. With this, they propose the conclave compiler that accelerates such queries by transforming them into a combination of data-parallel, local cleartext processing and MPC steps.  %TDB~\cite{maheshwari-osdi-2000} describes the design of a trusted database system on untrusted storage medium, and leverages a trusted storage medium to secure the database and its operations. 

% secure storage and query processing together in a single heterogeneous architecture.

% - djoin: privacy preserving join across multiple dbs
% - F. Emekci, D. Agrawal, A. E. Abbadi, and A. Gulbeden. 2006. Privacy Preserving Query Processing Using Third Parties. In ICDE. 27–27.: when data is residing at different parties; only acquire query result and nothing else; use third parties; reveal no extra info to third parties or data sources; hash based P2P system to select third parties to process queries, speeds up query resp time; P2P sys for anonymous comm and comp

% - private sql: https://dl.acm.org/doi/10.14778/3342263.3342274: differentially private answering of SQL queries with a fixed privacy budget; end-to-end differentially private relational db system; ensures a fixed privacy loss across all queries; support privacy policies on realistic schemas with multiple tables; aggregate queries; ensure differential privacy with a fixed privacy budget over all queries posed to the system; 
% -privacy integrated queries: PINQ; platform for privacy preserving data analysis; declarative programming lang in which all statements provide diff privacy; exec env to respect formal req of diff privacy; privacy provided by the platform itself; no source or derived data can return to the analysts; https://dl.acm.org/doi/10.1145/1559845.1559850
% - conclave: https://dl.acm.org/doi/10.1145/3302424.3303982: relational queries can can maintain MPC's end-to-end security guarantee without using crypto MPC techniques for all operations; conclave compiler accelerates such queries by transforming them into a combination of data-parallel, local cleartext processing and small MPC steps; MPC enabled query compiler makes MPC on big data efficient and accessible; turn query into combination of eficient, local processing steps and secure MPC steps;

%the context of secure NDP. More specifically, \project{} is the first secure NDP architecture, where it enables secure storage and secure split query execution across untrusted nodes over heterogenous ISA domains. %Our system design strives for strong freshness properties (rollback and forking attack protections), in addition to traditional confidentiality and integrity properties.


%Speicher~\cite{bailleu-fast-2019} presents the design of a secure storage system by creating a confidentiality preserving LSM data structure and enhance it to ensure data freshness. Systems such as SUNDR~\cite{li-osdi-2004}, Plutus~\cite{kallahalla-fast-2003}, jVPFS~\cite{weinhold-atc-2011}, SiRiUS~\cite{goh-ndss-2003}, SNAD~\cite{miller-fast-2002}, Maat~\cite{leung-sc-2007}, PCFS~\cite{garg-ieeesp-2010} present designs for securing data distributed across the network, over multiple nodes. Guardat~\cite{Vahldiek-Oberwagner-eurosys-2015} and Pesos~\cite{Krahn-eurosys-2018} enforce data access policies at the storage layer with the help of a small TCB. Strongbox~\cite{dickens-asplos-2018} presents the design of a full drive encryption engine that relies on stream ciphers to guarantee the confidentiality and integrity of data stored in the untrusted storage medium. 

% For distributed storage system design, Depot~\cite{Mahajan2011} and Salus~\cite{salus} provide secure distributed storage, which provide consistency, duarability, availablity and integrity.A2M~\cite{A2M} is also robust against Byzantine faults.\projecttitle{} additionally offers confidentiality.CloudProof~\cite{cloudproof} completely untrustes the cloud provider and is able to provide confidentiality, integrity serializablity and freshness in a distributed setting by leveraging existing distributed \gls{KV} stores.However, CloudProof requries the client to guarantee these security properties, which requires major changes to the client.In contrast to CloudProof, \project{} uses \gls{TEE} to provide the clients with the proofs for the security guarantees, alleviating the adoption barrier.




% Pesos~\cite{pesos},  Guardat~\cite{Vahldiek-Oberwagner2015}: file or object level policies that enforce access to data; present a small TCB at the storage layer that helps in enforcement;

% https://thoth.mpi-sws.org/
% Thoth: same stuff as above but kernel level monitor intercepting I/O; qapla: reference monitor with db adapter that intercepts queries and rewrites them to make them policy compliant

% https://www.gdprbench.org/publications : impact of gdpr policies on database systems(1,4); how gdpr is violated in current systems(2, 3,5);

% Exclaibur [USENIX Security] -- attestation: policy sealed data; centralized trusted monitor for scalability

% rodrigo, nuno, deepak's work.

% also gdpr related papers
%\myparagraph{Database access control}

%Further, database access control is an essential part of database systems as they prevent access from unauthorized parties to the database. Agrawal et al.~\cite{hippocratic-dbs} and LeFevre et al.~\cite{disclo-hippocratic-dbs} assume that data access control is enforced by the data owner at the granularity of a tuple, by introducing privacy policies at the data cell-level. Other access control methods use query rewriting techniques~\cite{ans-query-views, motro-access-auth}, propose extensions to SQL language~\cite{chaudhari-auth-predic-grant} and use authorization views~\cite{rizvi-query-rewrite-access-control}. Other techniques describe the use of query control~\cite{shay-acess-control-query-control} that provide query-based access control that restrict the queries that can be processed. Similar to some of these works in \project{}, we allow the user to specify data access policies which are enforced, later, by rewriting queries to restrict access to data on a per tuple basis.
Database access control prevents access from unauthorized parties to the database, and is essential. 
In ~\cite{hippocratic-dbs,disclo-hippocratic-dbs} data access control is enforced by the data owner, at the granularity of a tuple, by introducing access policies at the data cell-level. Other methods include
query rewriting techniques~\cite{ans-query-views, motro-access-auth}, extensions to SQL language~\cite{chaudhari-auth-predic-grant}, use authorization views~\cite{rizvi-query-rewrite-access-control}, and query control~\cite{shay-acess-control-query-control}. Similarly, % to some of these works 
in \project{} we let the user specify data access policies which are enforced, later, by rewriting queries to restrict access to data on a per tuple basis.

\myparagraph{Policy language and compliance}
Several policy languages can be used to restrict access to data and execution of queries~\cite{farnan-esorics-2011, PAQO-vldb-2013, mehta-security-2017, pesos-eurosys-2018, mehta-security-2017, beedkar-compl-geo-processing}.
%PAQO~\cite{farnan-esorics-2011, PAQO-vldb-2013} proposes extensions to SQL to declare policies in a declarative way that makes it possible to specify constraints on the kinds of plans that can be produced by the query optimizer to run distributed database queries. Guardat~\cite{guardat-eurosys-2015} and Pesos~\cite{pesos-eurosys-2018} describe a declarative language and API to define file and object level data access policies respectively. Qapla~\cite{mehta-security-2017} describes a SQL like policy specification language that can be used to tie policies to specific rows, columns or the querier's identity and time. The policies are a function of the database schema and are stored with the database itself.
% PAQO~\cite{farnan-esorics-2011, PAQO-vldb-2013} proposes extensions to SQL to declare policies, Guardat~\cite{guardat-eurosys-2015} and Pesos~\cite{pesos-eurosys-2018} introduce a declarative language and API to define data access policies, while Qapla~\cite{mehta-security-2017} describes an SQL-like policy specification language. Beedkar et al.~\cite{beedkar-compl-geo-processing} describe a query optimizer and declarative language to specify and enforce policies that restrict cross border movement of data. 
\project{} presents a new declarative policy language that allows users to describe data access and execution policies. %
% - https://dl.acm.org/doi/10.5555/2041225.2041270: specificication and enforcement of the querier's privacy contraints on the exec of distributed database queries; declarative way with extensions to SQL to specify constraints on the evaluation of the query;
% - https://dl.acm.org/doi/abs/10.14778/2536274.2536309?download=true; same as above; specify constraints on the kinds of plans that can be produced by the optimizer to run their queries;
% - guardat, pesos, qapla
%
Recent efforts~\cite{pesos-eurosys-2018, guardat-eurosys-2015, elnikety-security-2016, mehta-security-2017, garg-ieeesp-2010, schengendb-kraska, datumdb-kraska, policy-offload-vldb, privacy-op-offload-systor} enable clients to describe policies in a declarative language, a thin software layer at the object, file or query level ensures policy compliance data access. Further, attestation systems~\cite{santos-security-2012, palaemon-dsn-2020} provide for fast and scalable attestation.
\project{} builds upon prior work to create a unified abstraction to attest and policy compliance across the heterogeneous \csd.

% \if 0
% %\myparagraph{Attestation and policy compliance}
% Policy compliance for data access-accounting, usage, and processing must be strictly enforced under new government regulations (cf., GDPR). %, such as GDPR~\cite{shastricorr2019, mohanpoly2019, shastrihotcloud2019, gdpr-antipattern-cacm}. %Hence, policy-compliance in storage and data processing system is an active area of research~\cite{pesos-eurosys-2018, guardat-eurosys-2015, elnikety-security-2016, mehta-security-2017, garg-ieeesp-2010}. 
% Hence, several works~\cite{pesos-eurosys-2018, guardat-eurosys-2015, elnikety-security-2016, mehta-security-2017, garg-ieeesp-2010, schengendb-kraska, datumdb-kraska} enable clients to describe policies in a declarative language, a thin software layer at the object, file or query level ensures policy compliance data access. 
% %in a declarative language and separate the policy compliance from application execution by implementing it as a thin software layer either at the object, file or query level.
% %Kraska et al.~\cite{schengendb-kraska, datumdb-kraska} proposed an architectural vision for database systems that allows for auditing, deletion and user consent management.
% Other work~\cite{policy-offload-vldb, privacy-op-offload-systor} has explored enforcing compliance in distributed storage systems by offloading policy enforcement and privacy related operators closer to storage.
% Further, attestation systems~\cite{santos-security-2012, palaemon-dsn-2020} present new abstractions for fast and scalable attestation mechanisms.
% \project{} builds upon prior work to create a unified abstraction to attest and policy compliance across the heterogeneous \csd.
% % ADD http://vldb.org/pvldb/vol14/p1167-istvan.pdf https://dl.acm.org/doi/10.1145/3456727.3463769
% %As current systems are not built with policy compliance in mind, they suffer from performance degradation~\cite{shastrivldb2020, shahhotstorage2019}. \project{} proposes a trusted monitor that enforces client execution policies for the secure NDP architecture.  
% \fi